\documentclass[a4]{article}
\usepackage{gnuplottex}
\usepackage{csvsimple}
\usepackage{subcaption}
\usepackage{amsmath}
\usepackage{graphicx}
\usepackage{epstopdf}
\usepackage{float}

\graphicspath{ {./image/} }

\title{COMP26120 Lab 5 Report}
\author{Ziyi Li}
\begin{document}
\maketitle


\section{What makes a problem hard for Dynamic Programming?}

\subsection{Hypothesis}

When using a dynamic programming algorithm to solve the knapsack problem, there should be a positive relation between the backpack's size and the program's runtime, when the input data size is fixed. This means that the program takes more time to process when the knapsack size increases.

\subsection{Design}

The input data is generated by script provided (kp\_generate.py), in multiple groups. For the same size of backpack but different input data, I will run 3 groups of data with same backpack size. I hope the result will not variant. The range of size is between 1000 to 10000 with spacing 500. The program used here is dp\_kp.py, the the input data size(number of items) is fixed to 5000

\subsection{Results}
\subsection{Discussion}
\subsection{Data Statement}


\appendix

%% Any raw data or code scripts you want to present should be included as appendices.
\end{document}
